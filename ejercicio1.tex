\documentclass{article}

\usepackage[utf8]{inputenc}
\usepackage[spanish]{babel}
\usepackage{amsmath}

\begin{document}
Escribe la solución de la ecuacion en diferencias $x_{n+2}-4x_n=0$ con condiciones iniciales $x_0=1$ $x_1=-1$.

Solución:

Observemos que la ecuación resolvente es:$r^2-4=0$. Entonces se puede factorizar como $(r-2)(r+2)$, por lo cual las raíces son: $r_1=2$ $r_2=-2$
Es decir $X_n=a(2)^n+b(-2)^n$

Pero para que la solución safisfaga las condiciones iniciales:

$x_0=a+b=1$

$x_1=2a-2b=-1$

Entonces resolviendo el sistema de ecuaciones $a=\frac{1}{4}$ $b=\frac{3}{4}$
y sustituyendo los valores

  $$X_n=\frac{1}{4}(2)^n + \frac{3}{4}(-2)^n$$

\end{document}
