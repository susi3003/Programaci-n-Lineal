% Created 2019-04-05 vie 12:38
% Intended LaTeX compiler: pdflatex
\documentclass[11pt]{article}
\usepackage[utf8]{inputenc}
\usepackage[T1]{fontenc}
\usepackage{graphicx}
\usepackage{grffile}
\usepackage{longtable}
\usepackage{wrapfig}
\usepackage{rotating}
\usepackage[normalem]{ulem}
\usepackage{amsmath}
\usepackage{textcomp}
\usepackage{amssymb}
\usepackage{capt-of}
\usepackage{hyperref}
\author{Susana Lizbeth Rodriguez Espinosa}
\date{\today}
\title{Progamacion Lineal}
\hypersetup{
 pdfauthor={Susana Lizbeth Rodriguez Espinosa},
 pdftitle={Progamacion Lineal},
 pdfkeywords={},
 pdfsubject={},
 pdfcreator={Emacs 25.2.2 (Org mode 9.2.3)}, 
 pdflang={English}}
\begin{document}

\maketitle
\tableofcontents


\section{Teoría}
\label{sec:orgd32afd2}
\subsection{Motivación}
\label{sec:orgbe39da9}
El objetivo de la programación lineal es maximizar funciones lineales
sobre dominios convexos, es decir, definidos sobre regiones dadas por
desigualdades

\begin{center}
\includegraphics[width=.9\linewidth]{c.png}
\end{center}
\subsection{Ejemplos}
\label{sec:org898f252}

\begin{itemize}
\item El problema de la dieta
\item Optimización de lugares en una excursión
\item Escoger objetos Optimos para un campamento
\item El problema del flujo máximo
\end{itemize}

\subsection{Convexidad}
\label{sec:orgec9a3a6}
Un conjunto \(x\) es \textbf{Convexo} si para todos \(x,y\in X\) y \(t\in[0,1]\)

\subsection{Metodo Simplex}
\label{sec:org1c3f084}
\section{Herramientas computacionales}
\label{sec:orge615a45}
\subsection{Emacs}
\label{sec:org7917a3e}
\begin{center}
\begin{tabular}{ll}
C-x C-s & salvar archivo\\
C-x C-f & abrir archivo\\
M-q & formatear el párrafo\\
C-x d & editar directorios\\
C-g & salir de todo\\
C-x 2 & divide horizontalmente\\
C-x 3 & divide verticalmente\\
C-x 1 & regresa a una sola pantalla\\
M-w & copiar la región\\
C-w & borrar la región\\
Shif y flechas & seleccionar la región\\
C-y & pegar la región\\
\end{tabular}
\end{center}
\subsection{Git}
\label{sec:org959c919}
\begin{enumerate}
\item Git Hub
\label{sec:org666fff1}
\item Git en la terminal
\label{sec:orga84cf2d}
\end{enumerate}
\subsection{Python}
\label{sec:org7b7db00}
\begin{enumerate}
\item El lenguaje Python
\label{sec:org72d161e}
\item Jupyter
\label{sec:orgb9ca127}
\end{enumerate}
\subsection{\LaTeX{}}
\label{sec:orgc9d6be9}
\end{document}